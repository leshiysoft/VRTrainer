\documentclass[a4paper,12pt,oneside]{extreport}
%\documentclass[a4paper,12pt,oneside,draft]{extreport} % режим draft

\tolerance=9999 % терпимость к увеличенным межсловарным интервалам
\hbadness=9999 % выключить сообщения о плохости

%%%%%%%%%%%% настройка шрифтов %%%%%%%%%%%%%%%%%%%%%%%%%%%%%
\usepackage[english,russian]{babel} % перенос слов

% это для pdflatex
% \usepackage[utf8]{inputenc}
% \usepackage[T2A]{fontenc}

% это для lualatex
\usepackage{fontspec} % работа со шрифтами для LuaLaTeX
\setmainfont[Ligatures=TeX]{PT Serif} % выбор главного шрифта
\setsansfont[Ligatures=TeX]{PT Sans} % выбор sans шрифта
\setmonofont[Ligatures=TeX]{PT Mono} % выбор mono шрифта
%%%%%%%%%%%%%%%%%%%%%%%%%%%%%%%%%%%%%%%%%%%%%%%%%%%%%%%%%%%

\usepackage{luacode}

\usepackage{setspace} % межстрочный интервал
%\singlespacing
\onehalfspacing
%\doublespacing
%\setstretch{<factor>} % произвольный межстрочный интервал


%\setlength{\footnotesep}{0.5cm} % между отдельными сносками

\usepackage[textsize=footnotesize, textwidth=0.5cm]{todonotes} % Это для заметок на полях
\newcommand{\todoline}[1]{\todo[inline]{#1}}
\newcommand{\todom}{\todo{!}}

\usepackage[left=2cm,right=1.5cm,top=1.5cm,bottom=2cm]{geometry} % поля страницы

%\usepackage{amsmath} % Для формул
%\usepackage{amsfonts, amssymb} % Математические шрифты и символы
%\usepackage{amsthm} % окружения \newtheorem «теорема», «лемма» и т. п.

%\usepackage[final]{graphicx} % Включаем возможность вставки изображений

\usepackage{hyperref} % Включаем гиперссылки
\usepackage{xcolor} % Для расскраски ссылок
\definecolor{linkcolor}{HTML}{0000FF} % цвет ссылок
\definecolor{urlcolor}{HTML}{0000FF} % цвет гиперссылок
\hypersetup{pdfstartview=FitH,
			linkcolor=linkcolor,
			urlcolor=urlcolor,
			colorlinks=true} % Установка цветов для ссылок и гиперссылок

%\pagestyle{empty} % без номеров страниц
\pagestyle{plain} % с номерами страниц

\setlength{\topskip}{0pt} % Допустимый отступ от верхней границы листа
\usepackage{indentfirst} % Первая строка первого абзаца тоже с отступом
\setlength{\parindent}{1.25cm} % Отступ абзаца
\setlength{\parskip}{0.2cm} % Отступ между абзацами


\usepackage{enumitem}
%\setlist{nolistsep, itemsep=0.3cm,parsep=0pt}
\setlist{nolistsep} % убрать вертикальные отступы списков

%%%%%%%%%%%%%%%%%%% настройки нумерации %%%%%%%%%%%%%%%%%%%%%%%%%%%%%%%%
\renewcommand*\thesection{\arabic{section}}

\renewcommand{\labelenumii}{\theenumii.}
\renewcommand{\theenumii}{\theenumi.\arabic{enumii}}
\renewcommand{\labelenumiii}{\theenumiii.}
\renewcommand{\theenumiii}{\theenumii.\arabic{enumiii}}

\setcounter{tocdepth}{4}
\setcounter{secnumdepth}{4}
%%%%%%%%%%%%%%%%%%%%%%%%%%%%%%%%%%%%%%%%%%%%%%%%%%%%%%%%%%%%%%%%%%%%%%%


\begin{document}


\section{Структура директорий}

\begin{itemize}
	\item \textbf{models} --- 3D-модели используемые в тренажере, включающие текстуры, материалы (Substance) исходные файлы моделей (.blend), файлы моделей в формате Collada для Unity\footnote{особенность файлов моделей для Unity --- направление векторов: \textit{y-up, z-forward}} и файлы моделей в формате Collada для Substance\footnote{особенность файлов моделей для Substance --- направление векторов: \textit{z-up, y-forward} --- так же, как в Blender по умолчанию};
	\item \textbf{report} --- исходники настоящего отчета;
	\item \textbf{VRTrainer} --- Unity-проект тренажера виртуальной реальности.
\end{itemize}

Далее рассматривается структура отдельных деркторий проекта.

\subsection{Директория 3D моделей (models)}

\begin{itemize}
	\item \textbf{ModelsRoom} --- Unity-проект для тестирования 3D-моделей тренажера, представляет собой нечто похожее на проект-выставку или шоу-рум, если угодно;
	\item \textbf{references} --- директория с различными изображениями, которые были выбраны как основа (рефренс) для 3D-моделей и/или сцен, в тренажере использование этих изображений не предполагается; 
	\item \textbf{Roof} --- группа 3D-моделей для префабов потолочных секций;
	\item \textbf{Substance} --- файлы Substance с материалами, а также вспомогательными графами, для 3D-моделей;
	\item \textbf{Wall} --- группа 3D-моделей для префабов секций стен и проходов.
\end{itemize}

Директории с группами 3D-моделей, такие как Roof и Wall, имеют следующую структуру\footnote{некоторые необязательные директории могут отсутствовать}:

\begin{itemize}
	\item \textbf{blend-файл}, содержащий исходники моделей соответствующей группы;
	\item \textbf{директория meshes} --- файлы моделей группы в формате Collada для Unity;
	\item \textbf{директория sbs-meshes} (необязательная) --- файлы моделей группы в формате Collada для Substance;
	\item \textbf{директория textures} (необязательная) --- текстуры моделей группы. \todoline{Может вынести директорию \textbf{textures} в директорию \textbf{Substance}?}
\end{itemize}

\section{Структура Unity-проекта VRTrainer}
\todoline{...}

\section{Структура Unity-проекта ModeslRoom}
\todoline{...}

\end{document}




