\documentclass[a4paper,12pt,oneside]{extreport}
%\documentclass[a4paper,12pt,oneside,draft]{extreport} % режим draft

\tolerance=9999 % терпимость к увеличенным межсловарным интервалам
\hbadness=9999 % выключить сообщения о плохости

%%%%%%%%%%%% настройка шрифтов %%%%%%%%%%%%%%%%%%%%%%%%%%%%%
\usepackage[english,russian]{babel} % перенос слов

% это для pdflatex
% \usepackage[utf8]{inputenc}
% \usepackage[T2A]{fontenc}

% это для lualatex
\usepackage{fontspec} % работа со шрифтами для LuaLaTeX
\setmainfont[Ligatures=TeX]{PT Serif} % выбор главного шрифта
\setsansfont[Ligatures=TeX]{PT Sans} % выбор sans шрифта
\setmonofont[Ligatures=TeX]{PT Mono} % выбор mono шрифта
%%%%%%%%%%%%%%%%%%%%%%%%%%%%%%%%%%%%%%%%%%%%%%%%%%%%%%%%%%%

\usepackage{luacode}

\usepackage{setspace} % межстрочный интервал
%\singlespacing
\onehalfspacing
%\doublespacing
%\setstretch{<factor>} % произвольный межстрочный интервал


\usepackage[textsize=footnotesize]{todonotes} % Это для заметок на полях
%\setlength{\footnotesep}{0.5cm} % между отдельными заметками

\newcommand{\todoline}[1]{\todo[inline]{#1}}

\usepackage[left=2cm,right=1.5cm,top=1.5cm,bottom=2cm]{geometry} % поля страницы

%\usepackage{amsmath} % Для формул
%\usepackage{amsfonts, amssymb} % Математические шрифты и символы
%\usepackage{amsthm} % окружения \newtheorem «теорема», «лемма» и т. п.

%\usepackage[final]{graphicx} % Включаем возможность вставки изображений

\usepackage{hyperref} % Включаем гиперссылки
\usepackage{xcolor} % Для расскраски ссылок
\definecolor{linkcolor}{HTML}{0000FF} % цвет ссылок
\definecolor{urlcolor}{HTML}{0000FF} % цвет гиперссылок
\hypersetup{pdfstartview=FitH,
			linkcolor=linkcolor,
			urlcolor=urlcolor,
			colorlinks=true} % Установка цветов для ссылок и гиперссылок

%\pagestyle{empty} % без номеров страниц
\pagestyle{plain} % с номерами страниц

\setlength{\topskip}{0pt} % Допустимый отступ от верхней границы листа
\usepackage{indentfirst} % Первая строка первого абзаца тоже с отступом
\setlength{\parindent}{1.25cm} % Отступ абзаца
\setlength{\parskip}{0.2cm} % Отступ между абзацами


\usepackage{enumitem}
%\setlist{nolistsep, itemsep=0.3cm,parsep=0pt}
\setlist{nolistsep} % убрать вертикальные отступы списков

%%%%%%%%%%%%%%%%%%% настройки нумерации %%%%%%%%%%%%%%%%%%%%%%%%%%%%%%%%
\renewcommand*\thesection{\arabic{section}}

\renewcommand{\labelenumii}{\theenumii.}
\renewcommand{\theenumii}{\theenumi.\arabic{enumii}}
\renewcommand{\labelenumiii}{\theenumiii.}
\renewcommand{\theenumiii}{\theenumii.\arabic{enumiii}}

\setcounter{tocdepth}{4}
\setcounter{secnumdepth}{4}
%%%%%%%%%%%%%%%%%%%%%%%%%%%%%%%%%%%%%%%%%%%%%%%%%%%%%%%%%%%%%%%%%%%%%%%


\usepackage{dirtree}

\begin{document}


\section{Состав корневой директории проекта}

\dirtree{%
.1 .
.2 models.
.3 meshes.
.3 ModelsRoom.
.3 references.
.3 substance.
.4 materials.
.4 textures.
.2 report.
.2 VRTrainer.
}

\begin{itemize}
	\item \textbf{models} --- директория, содержащая 3d-модели используемые в тренажере, включая текстуры, материалы (графы Substance) исходные файлы моделей (.blend), файлы моделей в формате Collada для Unity\footnote{особенность файлов моделей для Unity --- направление векторов: \textit{y-up, z-forward}} и файлы моделей в формате Collada для Substance\footnote{особенность файлов моделей для Substance --- направление векторов: \textit{z-up, y-forward} --- так же, как в Blender по умолчанию};
	\item \textbf{models/ModelsRoom} --- директория, содеражащая Unity-проект для тестирования и отладки 3d-моделей тренажера;
	\item \textbf{models/references} --- директория с изображениями, которые были выбраны как основа (рефренс) для 3d-моделей и/или сцен, в тренажере использование этих изображений не предполагается; 
	\item \textbf{models/substance} --- директория, содержащая файлы Substance (графы материалов для 3d-моделей), а также текстуры, которые были сгенерированы с помощью Substance;
	\item \textbf{models/substance/materials} --- директория с файлами Substance;
	\item \textbf{models/substance/textures} --- директория, содержащая текстуры, которые были сгенерированы с помощью Substance;
	\item \textbf{models/meshes} --- директория с 3d-моделями;
	\item \textbf{report} --- исходники данного отчета;
	\item \textbf{VRTrainer} --- Unity-проект тренажера виртуальной реальности.
\end{itemize}


\section{3d-модели}

3d-модели содержатся в директории \textbf{models/meshes}. Некоторые 3d-модели объединены в группы. Даже если 3d-модель не объединена с другими в некоторую группу, то для этой модели создается отдельная группа. Каждая группа размещается в отдельной директории со следующей структурой: 

\dirtree{%
.1 .
.2 <имя файла>.blend.
.2 unity-meshes.
.2 sbs-meshes.
}

\begin{itemize}
	\item \textbf{<имя файла>.blend} --- файл с исходниками 3d-моделей;
	\item \textbf{unity-meshes} --- директория, содержащая файлы 3d-моделей в формате Collada для Unity;
	\item \textbf{sbs-meshes} --- директория, содержащая файлы 3d-моделей в формате Collada для Substance (директория необязательная и может отсутствовать);
	%\item \textbf{директория textures} (необязательная) --- текстуры моделей группы. \todoline{Может вынести директорию \textbf{textures} в директорию \textbf{Substance}?}
\end{itemize}

Директория \textbf{models/meshes} имеет иерархическую структуру --- несколько групп 3d-моделей объединены в одну общую группу, которая в свою очередь может быть объединена в другую общую группу и так далее. 


\section{Текстуры сгенерированные с помощью Substance}

Cгенирированные на основе Substance материалов текстуры содержатся в директории \textbf{models/substance/texture}. Эта директория имеет иерархическую структуру --- текстуры объединяются в группы точно также как и 3d-модели, но на каждом уровне иерархии в директории \textbf{common} могут содержаться текстуры, которые являются общими для нескольких подгрупп данной группы.


\section{Структура Unity-проекта VRTrainer}
\todoline{...}

\section{Структура Unity-проекта ModeslRoom}
\todoline{...}

\section{Термины и сокращения}
\todoline{...}

\end{document}




